%% ***************************************************************
%  Copyright (C) Luca Parolari 2020
%  
%  luca.parolari23@gmail.com
%
%  You should have received a copy of the license with this file,
%  if not write to the author and request the license.

% !TeX spellcheck = it_IT

%\documentclass[addpoints,12pt]{exam}
\documentclass[addpoints,12pt,answers]{exam}

%% ***************************************************************
%  PACKAGES
%  ========
\input{packages.tex}

%% ***************************************************************
%  RESOURCES
%  =========
\input{prooftree.tex}
\input{macros.tex}

%% ***************************************************************
%  CONFIGURATIONS
%  ==============
\input{exerciseconfig.tex}


% ****************************************************************
% DOCUMENT
% ========

\author{Luca Parolari\footnote{\href{mailto:luca.parolari23@gmail.com}{luca.parolari23@gmail.com}}}

\begin{document}
    
    \title{Ricerca Date}
    \date{Febbraio 2019}
    
    \maketitle
    
    Leggere attentamente la consegna e svolgere l'esercizio.
    
    \section{Consegna}
    
    \begin{itemize}
    	\item Sia \texttt{Data} un tipo \emph{struct} costituito da tre campi, $g, m, a$, di tipo \texttt{int}. Scrivere una funzione booleana di nome ricerca che, presi come suoi parametri un array $v$ di puntatori a \texttt{Data}, il numero $n$ di elementi in $v$ , ed una struttura d di tipo \texttt{Data}, restituisce \texttt{true} se $v$ contiene $d$, \texttt{false} altrimenti.
    	\item Scrivere anche un main di prova che legge da standard input una sequenza di date (max 100), nel formato $g m a$, le memorizza una alla volta in un array $A$ di puntatori a \texttt{Data} e quindi verifica, utilizzando (obbligatoriamente) la funzione ricerca, se $A$ contiene la data \texttt{15/8/1994}. La lettura termina non appena viene letta una data con giorno 0.
    \end{itemize}

	\emph{Hint.} Memorizzare una data in $A$ significa allocare una nuova struttura Data per contenere la data letta e quindi memorizzare in $A$ il puntatore a questa struttura.
    
\end{document}