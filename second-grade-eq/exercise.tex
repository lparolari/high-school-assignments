%% ***************************************************************
%  Copyright (C) Luca Parolari 2020
%  
%  luca.parolari23@gmail.com
%
%  You should have received a copy of the license with this file,
%  if not write to the author and request the license.

% !TeX spellcheck = it_IT

%\documentclass[addpoints,12pt]{exam}
\documentclass[addpoints,12pt,answers]{exam}

%% ***************************************************************
%  PACKAGES
%  ========
\input{packages.tex}

%% ***************************************************************
%  RESOURCES
%  =========
\input{prooftree.tex}
\input{macros.tex}

%% ***************************************************************
%  CONFIGURATIONS
%  ==============
\input{exerciseconfig.tex}


% ****************************************************************
% DOCUMENT
% ========

\author{Luca Parolari\footnote{\href{mailto:luca.parolari23@gmail.com}{luca.parolari23@gmail.com}}}

\begin{document}
    
    \title{Equazioni di secondo grado}
    \date{Dicembre 2019}
    
    \maketitle
    
    Leggere attentamente la consegna e svolgere l'esercizio.
    
    \section{Consegna}
    
    Scrivere un programma che calcola le radici di un'equazione di secondo grado
    \[ a x^2 + b x + c = 0 \]
    I coefficienti dell'equazione sono letti da standard input. Se il discriminante è negativo il programma stampa un opportuno messaggio e quindi termina. Controllare e distinguere anche il caso di discriminante $= 0$ (soluzioni coincidenti).
    
    \textit{Hint.} Per il calcolo della radice quadrata di un numero n utilizzare la funzione \texttt{sqrt(n)} fornita dalla libreria 
    \texttt{cmath} (richiede di aggiungere la direttiva \verb|#include <cmath>|).
    
    \section{Esempi di utilizzo} \footnote{Input sottolineati.}

	\noindent\textbf{Esempio}
	\begin{lstlisting}[style=verbatim]
	Risoluzione dell'equazione di II grado
	    a x^2 + b x + c = 0
	Inserisci i coefficienti dell'equazione:
	a = %\underline{1}%
	b = %\underline{-5}%
	c = %\underline{4}%
	Soluzioni:
	x_1 = 4
	x_2 = 1
	\end{lstlisting}

	\noindent\textbf{Esempio}
	\begin{lstlisting}[style=verbatim]
	Risoluzione dell'equazione di II grado
	    a x^2 + b x + c = 0
	Inserisci i coefficienti dell'equazione:
	a = %\underline{4}%
	b = %\underline{-4}%
	c = %\underline{1}%
	Soluzioni:
	x_1 = x_2 = 0.5
	\end{lstlisting}
	
	\noindent\textbf{Esempio}
	\begin{lstlisting}[style=verbatim]
	Risoluzione dell'equazione di II grado
	    a x^2 + b x + c = 0
	Inserisci i coefficienti dell'equazione:
	a = %\underline{3}%
	b = %\underline{-5}%
	c = %\underline{4}%
	Discriminante negativo!
	\end{lstlisting}

\end{document}