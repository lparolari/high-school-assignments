%% ***************************************************************
%  Copyright (C) Luca Parolari 2020
%  
%  luca.parolari23@gmail.com
%
%  You should have received a copy of the license with this file,
%  if not write to the author and request the license.

% !TeX spellcheck = it_IT

%\documentclass[addpoints,12pt]{exam}
\documentclass[addpoints,12pt,answers]{exam}

%% ***************************************************************
%  PACKAGES
%  ========
\input{packages.tex}

%% ***************************************************************
%  RESOURCES
%  =========
\input{prooftree.tex}
\input{macros.tex}

%% ***************************************************************
%  CONFIGURATIONS
%  ==============
\input{exerciseconfig.tex}


% ****************************************************************
% DOCUMENT
% ========

\author{Luca Parolari\footnote{\href{mailto:luca.parolari23@gmail.com}{luca.parolari23@gmail.com}}}

\begin{document}
    
    \title{Stampa rettangolo}
    \date{Dicembre 2019}
    
    \maketitle
    
    Leggere attentamente la consegna e svolgere l'esercizio.
    
    \section{Consegna}
    
    Scrivere un programma che stampi un rettangolo la cui cornice sia costituita da caratteri asterisco e la parte interna da un carattere immesso dall'utente. Il numero di righe e di colonne è specificato
    dall'utente tramite standard input (accettare soltanto valori $> 0$).
    
    \section{Esempi di utilizzo}
	\footnote{Input sottolineato.}

	\begin{lstlisting}[style=verbatim]
		Inserisci quante righe vuoi: %\underline{10}%
		Inserisci quante colonne vuoi: %\underline{30}%
		Inserisci il carattere che riempira' il rettangolo: %\underline{.}%
		
		******************************
		*............................*
		*............................*
		*............................*
		*............................*
		*............................*
		*............................*
		*............................*
		*............................*
		******************************
	\end{lstlisting}
    
\end{document}