%%% CONFIGURATIONS FOR EXERCISE
% Customized by Luca Parolari
% luca.parolari23@gmail.com

%% tikz: flowchart
\usetikzlibrary{shapes,arrows,arrows.meta}
\tikzset{%
	>={Latex[width=2mm,length=2mm]},
	% Specifications for style of nodes:
	base/.style = {rectangle, rounded corners, draw=black,
		minimum width=4cm, minimum height=1cm,
		text centered, font=\sffamily},
	cloud/.style = {draw, ellipse, fill=white},%red!20},
	decision/.style = {diamond, draw, fill=white},%blue!20},
	process/.style = {base, minimum width=2.5cm, fill=white,%blue!20,
		font=\ttfamily},
}

%% lstlisting
\lstset{autogobble=true}

\lstdefinestyle{mycpp}{
	basicstyle=\ttfamily,
	breaklines=true,
	%captionpos=b,                    % sets the caption-position to bottom
	%deletekeywords={...},            % if you want to delete keywords from the given language
	escapeinside={\%*}{*)},          % if you want to add LaTeX within your code
	frame=l,                    % adds a frame around the code
	%keepspaces=true,                 % keeps spaces in text, useful for keeping indentation of code (possibly needs columns=flexible)
	language=C++,
	%morekeywords={*,...},            % if you want to add more keywords to the set
	numbers=left,                    % where to put the line-numbers; possible values are (none, left, right)
	numbersep=5pt,                   % how far the line-numbers are from the code
	showspaces=false,                % show spaces everywhere adding particular underscores; it overrides 'showstringspaces'
	showstringspaces=false,          % underline spaces within strings only
	showtabs=false,                  % show tabs within strings adding particular underscores
}

\lstdefinestyle{mycppbox}{
	basicstyle=\ttfamily,
	breaklines=true,
	%captionpos=b,                    % sets the caption-position to bottom
	%deletekeywords={...},            % if you want to delete keywords from the given language
	escapeinside={\%*}{*)},          % if you want to add LaTeX within your code
	frame=single,                    % adds a frame around the code
	%keepspaces=true,                 % keeps spaces in text, useful for keeping indentation of code (possibly needs columns=flexible)
	language=C++,
	%morekeywords={*,...},            % if you want to add more keywords to the set
	numbers=left,                    % where to put the line-numbers; possible values are (none, left, right)
	numbersep=5pt,                   % how far the line-numbers are from the code
	showspaces=false,                % show spaces everywhere adding particular underscores; it overrides 'showstringspaces'
	showstringspaces=false,          % underline spaces within strings only
	showtabs=false,                  % show tabs within strings adding particular underscores
}

\lstdefinestyle{verbatim}{
	basicstyle=\ttfamily,
	columns=fixed,
	fontadjust=true,
	basewidth=0.5em,
	escapechar=\%
}