%% ***************************************************************
%  Copyright (C) Luca Parolari 2020
%  
%  luca.parolari23@gmail.com
%
%  You should have received a copy of the license with this file,
%  if not write to the author and request the license.

% !TeX spellcheck = it_IT

%\documentclass[addpoints,12pt]{exam}
\documentclass[addpoints,12pt,answers]{exam}

%% ***************************************************************
%  PACKAGES
%  ========
\input{packages.tex}

%% ***************************************************************
%  RESOURCES
%  =========
\input{prooftree.tex}
\input{macros.tex}
\input{config.tex}

%% ***************************************************************
%  CONFIGURATIONS
%  ==============
\input{exerciseconfig.tex}


% ****************************************************************
% DOCUMENT
% ========

\author{Luca Parolari\footnote{\href{mailto:luca.parolari23@gmail.com}{luca.parolari23@gmail.com}}}

\begin{document}
    
    \title{Controllo Array Ordinato}
    \date{Data}
    
    \maketitle
    
    Leggere attentamente la consegna e svolgere l'esercizio.
    
    \section{Consegna}
    
    Realizzare un programma in linguaggio C++ che riceve in input una sequenza di $N$ numeri interi, con $N$ dato da utente. I numeri sono memorizzati in un vettore, ed il vettore può contenere al massimo $30$ numeri. Terminato l’inserimento della sequenza di numeri, il programma deve svolgere le seguenti operazioni.
    \begin{itemize}
        \item \textbf{Verificare} se il vettore contiene una sequenza di numeri ordinata in modo crescente e mostrare un messaggio all'utente che indica lo stato dell'array;
        \item \textbf{Rovesciare} il vettore nel caso esso sia ordinato in modo crescente. 
    \end{itemize}
    Le funzionalità elencate devono essere realizzate tramite funzioni apposite.
    
    \section{Esempi di utilizzo}
    \footnote{Input sottolineato.}
    
    \noindent\textbf{Esempio}
	\begin{lstlisting}[style=verbatim]
        Quanti numeri vuoi inserire? %\underline{5}%
        Dai i numeri: %\underline{4 3 9 8 7}%

        La sequenza di numeri non %\`e% ordinata.
	\end{lstlisting}

    \noindent\textbf{Esempio}
    \begin{lstlisting}[style=verbatim]
        Quanti numeri vuoi inserire? %\underline{5}%
        Dai i numeri: %\underline{4 4 5 8 17}%
		
        La sequenza di numeri %\`e% ordinata.
        L'array rovesciato %\`e%: [17, 8, 5, 4, 4]
	\end{lstlisting}
    
\end{document}