%% ***************************************************************
%  Copyright (C) Luca Parolari 2020
%  
%  luca.parolari23@gmail.com
%
%  You should have received a copy of the license with this file,
%  if not write to the author and request the license.

% !TeX spellcheck = it_IT

%\documentclass[addpoints,12pt]{exam}
\documentclass[addpoints,12pt,answers]{exam}

%% ***************************************************************
%  PACKAGES
%  ========
\input{packages.tex}

%% ***************************************************************
%  RESOURCES
%  =========
\input{prooftree.tex}
\input{macros.tex}

%% ***************************************************************
%  CONFIGURATIONS
%  ==============
\input{exerciseconfig.tex}


% ****************************************************************
% DOCUMENT
% ========

\author{Luca Parolari\footnote{\href{mailto:luca.parolari23@gmail.com}{luca.parolari23@gmail.com}}}

\begin{document}
    
    \title{Titolo\\ \large Sottotitolo}
    \date{Data}
    
    \maketitle
    
    Leggere attentamente la consegna e svolgere l'esercizio.
    
    \section{Consegna}
    
    Scrivere un programma che legge da un file di nome \texttt{dati.txt} una sequenza di numeri interi, la memorizza in un array dinamico $D$ di dimensione iniziale $10$; ordina l'array $D$ in senso crescente utilizzando una funzione \texttt{ordina} e quindi memorizza l'array ordinato sullo stesso file \texttt{dati.txt}. Nel caso in cui l'array $D$ non sia sufficiente a contenere i dati letti dal file, il programma provvede ad aumentare la capacità di $D$ utilizzando una funzione \texttt{raddoppia} che, presi come suoi parametri un array e la sua dimensione, raddoppia la dimensione dell'area riservata all'array.
    
\end{document}