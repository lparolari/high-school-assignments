%% ***************************************************************
%  Copyright (C) Luca Parolari 2020
%  
%  luca.parolari23@gmail.com
%
%  You should have received a copy of the license with this file,
%  if not write to the author and request the license.

% !TeX spellcheck = it_IT

%\documentclass[addpoints,12pt]{exam}
\documentclass[addpoints,12pt,answers]{exam}

%% ***************************************************************
%  PACKAGES
%  ========
\input{packages.tex}

%% ***************************************************************
%  RESOURCES
%  =========
\input{prooftree.tex}
\input{macros.tex}
\input{config.tex}

%% ***************************************************************
%  CONFIGURATIONS
%  ==============
\input{exerciseconfig.tex}


% ****************************************************************
% DOCUMENT
% ========

\author{Luca Parolari\footnote{\href{mailto:luca.parolari23@gmail.com}{luca.parolari23@gmail.com}}}

\begin{document}
    
    \title{Geometric Drawer}
    \date{30/03/2020}
    
    \maketitle
    
    Leggere attentamente la consegna e svolgere l'esercizio.
    
    \section{Consegna}
    Si chiede di realizzare il seguente programma, che consente di disegnare sulla console un insieme predefinito di figure geometriche.

    Il main, fornito in allegato alla consegna ed estendibile, crea alcune figure che memorizza in un array e successivamente, scorrendo l'array stampa invoca sulle figure il metodo \texttt{draw()}, per disegnarle.

    Si noti che viene sfruttato il polimorfismo, che consente alle figure di essere figure specializzate (es. Linee, Cerchi, Rettangoli, ...) e quindi con possibilità di essere concretamente disegnate.

    Realizzare di conseguenza le seguenti classi:
    \begin{itemize}
        \item \texttt{Figure}\footnote{Attenzione: la parola è scritta in inglese e ha la lettera F maiuscola.}, che rappresenta l'interfaccia di tutte le figure che il nostro programma sa disegnare. La classe offre un unico metodo astratto puro (ovvero \texttt{virtual} e senza implementazione (\texttt{= 0})) che restituisce \texttt{void}.

        \item \texttt{Circle}, che rappresenta un cerchio. La classe mette a disposizione un costruttore che prende in input \texttt{x} e \texttt{y} le coordinate del centro del cerchio, ed il raggio. Questa classe deve implementare in modo corretto il metodo \texttt{draw()} per disegnare a video il cerchio.\\
        \textbf{Attenzione:} la classe deve ereditare le proprietà dell'interfaccia \texttt{Figure}.

        \item \texttt{HorizontalSegment}, che rappresenta un segmento orizzontale. La classe mette a disposizione un costruttore che prende in input \texttt{x} e \texttt{y} le coordinate che rappresentano l'inizio del segmento, e la sua lunghezza. Questa classe deve implementare in modo corretto il metodo \texttt{draw()} per disegnare a video il segmento.\\
        \textbf{Attenzione:} la classe deve ereditare le proprietà dell'interfaccia \texttt{Figure}.
    \end{itemize}

    \vspace*{1em}

    \textbf{Suggerimento}: si veda la console come un piano cartesiano, in cui l'asse y è comunemente orientato da sinistra a destra, mentre l'asse x è orientato dall'alto verso il basso. L'origine degli assi è considerata, per ogni figura, come il primo spazio disponibile in alto a sinistra nella console.

    \vspace*{1em}

    \textbf{Suggerimento}: per disegnare il cerchio pieno (vedi allegato), si utilizzi la disequazione $(x-x_c)^2 + (y-y_c)^2 \leq r^2$ dove $x_c, y_c$ sono le coordinate del centro del cerchio e $r$ è il raggio del cerchio, che identifica l'area interna del cerchio. Quindi, se si sostituisce un punto di coordinate $i$ e $j$ ad $x$ e $y$ nella disequazione precedente ed essa risulta vera allora il punto è da disegnare, altrimenti inserire della spaziatura.

    \vspace*{1em}

    \textbf{Suggerimento}: per mantenere le proporzioni delle figure è necessario aumentare lo spazio tra i caratteri stampati sull'asse y. Ad esempio, se si vuole stampare un punto per rappresentare una linea, non stampare solamente asterisco (\texttt{cout << "*"}), stampare un asterisco seguito da uno spazio (\texttt{cout << "* "}). Questo torna utile soprattutto per disegnare il cerchio.
    
    \section{Esempi}
    
	\begin{lstlisting}[style=verbatim]
        > ./a.out
                      *               
                * * * * * * *         
              * * * * * * * * *       
            * * * * * * * * * * *     
          * * * * * * * * * * * * *   
          * * * * * * * * * * * * *   
          * * * * * * * * * * * * *   
        * * * * * * * * * * * * * * * 
          * * * * * * * * * * * * *   
          * * * * * * * * * * * * *   
          * * * * * * * * * * * * *   
            * * * * * * * * * * *     
              * * * * * * * * *       
                * * * * * * *         
                      *               
        ------------------------
        * * * * * * * * 
        * * * * * * *   
        * * * * * * *   
        * * * * * * *   
        * * * * * *     
        * * * * *       
        * * * *         
        *               
        ------------------------
        
          * * * * * 
        ------------------------
	\end{lstlisting}
    
    \section{Allegati}

    \subsection{Main}

    \begin{lstlisting}[style=mycpp]
        int main() {
            Circle* c1 = new Circle(7, 7, 7);
            Circle* c2 = new Circle(0, 0, 7);
            HorizontalSegment* s1 = new HorizontalSegment(1, 1, 5);
        
            vector<Figure*> ff;
            ff.push_back(c1);
            ff.push_back(c2);
            ff.push_back(s1);
        
            for (Figure* f : ff) {
                f->draw();
                cout << "-------------------------------------------------" << endl;
            }
        
            return 0;
        }
    \end{lstlisting}

\end{document}