%% ***************************************************************
%  Copyright (C) Luca Parolari 2020
%  
%  luca.parolari23@gmail.com
%
%  You should have received a copy of the license with this file,
%  if not write to the author and request the license.

% !TeX spellcheck = it_IT

%\documentclass[addpoints,12pt]{exam}
\documentclass[addpoints,12pt,answers]{exam}

%% ***************************************************************
%  PACKAGES
%  ========
\input{packages.tex}

%% ***************************************************************
%  RESOURCES
%  =========
\input{prooftree.tex}
\input{macros.tex}
\input{config.tex}

%% ***************************************************************
%  CONFIGURATIONS
%  ==============
\input{exerciseconfig.tex}


% ****************************************************************
% DOCUMENT
% ========

\author{Luca Parolari\footnote{\href{mailto:luca.parolari23@gmail.com}{luca.parolari23@gmail.com}}}

\begin{document}

\title{Metric Conversions}
\date{AS 2019-2020}

\maketitle

Leggere attentamente la consegna e svolgere l'esercizio.

\section{Consegna}

Progettare una classe \texttt{Converter} che ha il compito di convertire
le unità di misura: metri a chilometri, celsius a fahrenheit, byte a
kilobyte, eccetera.

La classe Converter viene istanziata specificando il tipo di conversione,
per esempio \texttt{Converter c1("metri\textunderscore kilometri")} oppure
\texttt{Converter c2("celsius\textunderscore fahrenheit")}.

La classe Converter offre un metodo \texttt{converti(numero)} che restituisce
in base a quanto specificato il valore convertito.

Per esempio:
\begin{lstlisting}[style=mycpp]
Converter c1("metri_kilometri");
c1.converti(1000) // restituisce 1
c1.converti(2000) // restituisce 2

Converter c2("celsius_fahrenheit")
c2.converti(20) // restituisce 68
\end{lstlisting}

Testare nel main i vari convertitori implementati.

Provare ad estendere la classe per renderla il più generica possibile,
ovvero: nel costruttore specificare la dimensione sulla quale convertire
e nel metodo le unità di misura.

Per esempio
\begin{lstlisting}[style=mycpp]
    Converter lunghezze("lunghezze");
    lunghezze.converti("metri", "chilometri", 1000); // restituisce 1
    lunghezze.converti("centimetri", "metri", 100); // restituisce 1
    \end{lstlisting}

Prendere esempio dal convertitore di google: \href{http://tiny.cc/cebokz}{http://tiny.cc/cebokz}

\pagebreak

\section{Estensione: Ereditarieta}
\footnote{Estensione obbligatoria}

Estendere l'esercizio precedente con l'ereditarietà: si definisca \texttt{Converter}
come una classe astratta, indicando il prototipo del metodo \texttt{converti(...)}.
Realizzare poi delle sottoclassi di \texttt{Converter} che implementino alcuni convertitori
come quelli sulle unità di misura dell'esercizio precedente ma anche convertitori più
generali come \texttt{DoubleConverter} che restituisce l'input raddoppiato,
\texttt{SquareConverter} che restituisce $input^2$, eccetera.

Modificare il main aggiungendo un vettore (o un array) di convertitori e dato un input
(per es. numero intero) stampare a video tutte le conversioni dell'input date dai
convertitori inseriti nel vettore.

\end{document}