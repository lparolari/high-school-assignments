%% ***************************************************************
%  Copyright (C) Luca Parolari 2020
%  
%  luca.parolari23@gmail.com
%
%  You should have received a copy of the license with this file,
%  if not write to the author and request the license.

% !TeX spellcheck = it_IT

%\documentclass[addpoints,12pt]{exam}
\documentclass[addpoints,12pt,answers]{exam}

%% ***************************************************************
%  PACKAGES
%  ========
\input{packages.tex}

%% ***************************************************************
%  RESOURCES
%  =========
\input{prooftree.tex}
\input{macros.tex}
\input{config.tex}

%% ***************************************************************
%  CONFIGURATIONS
%  ==============
\input{exerciseconfig.tex}


% ****************************************************************
% DOCUMENT
% ========

\author{Luca Parolari\footnote{\href{mailto:luca.parolari23@gmail.com}{luca.parolari23@gmail.com}}}

\begin{document}

\title{Num-Words}
\date{AS 2019-2020}

\maketitle

Leggere attentamente la consegna e svolgere l'esercizio.

\section{Consegna}

Progettare e realizzare una funzione che dato un numero in input,
stampa a video la sua rappresentazione in parola. La funzione può
essere limitata solo a certi input: è possibile decidere di convertire
solo numeri inferiori ad una certa soglia (e.g., $\leq 999999$).

\begin{lstlisting}[style=mycpp]
converti(1) // stampa "uno"
converti(1167) // stampa "millecentosessantasette"
\end{lstlisting}

\'E possibile restituire il valore al posto che stamparlo a video se ritenuto più comodo.

\textbf{Attenzione}: non è obbligatorio realizzare la funzione ricorsiva,
ma potrebbe essere molto complicato trovarne una iterativa: si ragioni
sulla struttura della soluzione.

Per esempio, è possibile vedere la rappresentazione in lettere del numero 2822 come
\begin{align*}
     & \text{due} + \text{mila} + (\text{\textit{rappresentazione di 822}})                                               \\
     & = \text{due} + \text{mila} + (\text{otto} + \text{cento} + \text{\textit{rappresentazione di 22}})                 \\
     & = \text{due} + \text{mila} + (\text{otto} + \text{cento} + (\text{venti} + \text{\textit{rappresentazione di 2}})) \\
     & = \text{due} + \text{mila} + (\text{otto} + \text{cento} + (\text{venti} + (\text{due})))
\end{align*}

Inoltre, bisogna prestare attenzione ai casi particolari, per esempio 1117 non è composto da
$\text{uno} + \text{mila} + \text{\textit{rappresentazione di 117}}$, ma da
$\text{mille} + \text{\textit{rappresentazione di 117}}$, così come 17 non è
rappresentato da $\text{dieci} + \text{sette}$.

Cercare quindi di trovare più regole possibili per evitare di
riscrivere molte volte la stessa parte di codice.

Realizzare infine un main che alcune conversioni, specialmente quelle più critiche.
\'E possibile, opzionalmente, fornire interazione con l'utente chiedendo in input
il numero da convertire.

\section{Estensione*}

Realizzare la funzione che svolge la conversione inversa rispetto a quella descritta sopra,
ovvero prende in input una stringa che rappresenta un numero in parole e ne restituisce
l'intero corrispondente.

\end{document}