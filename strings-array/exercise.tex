%% ***************************************************************
%  Copyright (C) Luca Parolari 2020
%  
%  luca.parolari23@gmail.com
%
%  You should have received a copy of the license with this file,
%  if not write to the author and request the license.

% !TeX spellcheck = it_IT

%\documentclass[addpoints,12pt]{exam}
\documentclass[addpoints,12pt,answers]{exam}

%% ***************************************************************
%  PACKAGES
%  ========
\input{packages.tex}

%% ***************************************************************
%  RESOURCES
%  =========
\input{prooftree.tex}
\input{macros.tex}

%% ***************************************************************
%  CONFIGURATIONS
%  ==============
\input{exerciseconfig.tex}


% ****************************************************************
% DOCUMENT
% ========

\author{Luca Parolari\footnote{\href{mailto:luca.parolari23@gmail.com}{luca.parolari23@gmail.com}}}

\begin{document}
    
    \title{Array di Stringhe}
    \date{Gennaio 2020}
    
    \maketitle
    
    Leggere attentamente la consegna e svolgere l'esercizio.
    
    \section{Consegna}
    
    Scrivere un programma principale che legge da std input una sequenza di stringhe (ciascuna di lunghezza massima 100 cararatteri), fino a incontrare la stringa \texttt{"stop"}, e le memorizza una alla volta in un array $A$ di stringhe (dimensione massima 1000) nel modo seguente: per ogni stringa letta $x$, se $x$ non è ancora presente in $A$, alloca la memoria per la nuova stringa e aggiunge in fondo ad A il puntatore alla stringa allocata; se $x$ è già presente in $A$, aggiunge in fondo ad $A$ il puntatore alla stringa trovata (senza allocare nuova memoria per $x$).
    Al termine il programma stampa il numero di stringhe diverse memorizzate in $A$ e la sequenza completa delle stringhe lette.
    
    \textit{Nota}. Utilizzare soltanto stringhe ``tipo C''.
    
    \section{Esempi di utilizzo}
    \footnote{Input sottolineato.}
    
	\begin{lstlisting}[style=verbatim]
		Dai una sequenza di stringhe ("stop" per smettere):
		%\underline{il gatto mangia il topo ma il topo scappa stop}%
		
		Memorizzate 6 stringhe diverse
		Sequenza: il gatto mangia il topo ma il topo scappa
	\end{lstlisting}
    
\end{document}