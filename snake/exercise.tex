%% ***************************************************************
%  Copyright (C) Luca Parolari 2020
%  
%  luca.parolari23@gmail.com
%
%  You should have received a copy of the license with this file,
%  if not write to the author and request the license.

% !TeX spellcheck = it_IT

%\documentclass[addpoints,12pt]{exam}
\documentclass[addpoints,12pt,answers]{exam}

%% ***************************************************************
%  PACKAGES
%  ========
\input{packages.tex}

%% ***************************************************************
%  RESOURCES
%  =========
\input{prooftree.tex}
\input{macros.tex}

%% ***************************************************************
%  CONFIGURATIONS
%  ==============
\input{exerciseconfig.tex}


% ****************************************************************
% DOCUMENT
% ========

\author{Luca Parolari\footnote{\href{mailto:luca.parolari23@gmail.com}{luca.parolari23@gmail.com}}}

\begin{document}
    
    \title{Snake}
    \date{Gennaio 2020}
    
    \maketitle
    
    Leggere attentamente la consegna e svolgere l'esercizio.
    
    \section{Prima versione}
    
  	Viene fornito un file con funzionalità di base da cui partire per l'implementazione della logica di Snake vero e proprio. (Vedere il capitolo \ref{ch_res}).
    
    \subsection{Consegna}
    
    La prima versione del gioco Snake, molto semplice, presenta le seguenti caratteristiche.
    \begin{itemize}
    	\item Una matrice rappresenta la mappa di gioco.
    	\item Un carattere, ad esempio \texttt{@}, rappresenta la testa dello Snake (ovvero, lo Snake ha lunghezza 1).
    	\item Lo Snake si muove per iterazioni successive: ad ogni iterazione viene mostrata la matrice all'utente, successivamente viene chiesto all'utente in che direzione tra destra, sinistra, su, giù muovere lo Snake (ad esempio, leggendo da standard input un carattere e facendo la mossa se esso corrisponde a uno tra \texttt{w}, \texttt{a}, \texttt{s}, \texttt{d}).
    	\item Lo Snake muore se esce dalla mappa.
    \end{itemize}


	\subsection{Esempi di utilizzo}
	\footnote{Input sottolineato.}
	
	\begin{lstlisting}[style=verbatim]
		> ./a.out 
		# # # # # # 
		# @       #
		#         #
		#         #
		#         #
		# # # # # # 
		
		Prossima posizione: %\underline{s}%
		# # # # # # 
		#         #
		# @       #
		#         #
		#         #
		# # # # # # 
		
		Prossima posizione: %\underline{d}%
		# # # # # # 
		#         #
		#   @     #
		#         #
		#         #
		# # # # # # 
		
		Prossima posizione: %\underline{d}%
		# # # # # # 
		#         #
		#     @   #
		#         #
		#         #
		# # # # # #
		
		Prossima posizione: %\underline{d}%
		# # # # # # 
		#         #
		#       @ #
		#         #
		#         #
		# # # # # #  
		
		Prossima posizione: %\underline{d}%
		Game over!
	\end{lstlisting}

	\section{Seconda versione}
	
	\subsection{Consegna}
	
	\begin{itemize}
		\item Aggiungere ``l'effetto pacman'': se lo Snake sbatte contro un muro, esso ricompare dall'altra parte della mappa. Per esempio, lo Snake sbatte contro il muro a destra, la testa dello Snake ricompare sulla sinistra della mappa.
	\end{itemize}
    
    \subsection{Esempi di utilizzo}
    \footnote{Input sottolineato.}
    
	\begin{lstlisting}[style=verbatim]
		> ./a.out 
		# # # # # # 
		# @       #
		#         #
		#         #
		#         #
		# # # # # # 
		
		Prossima posizione: %\underline{s}%
		# # # # # # 
		#         #
		# @       #
		#         #
		#         #
		# # # # # # 
		
		Prossima posizione: %\underline{d}%
		# # # # # # 
		#         #
		#   @     #
		#         #
		#         #
		# # # # # # 
		
		Prossima posizione: %\underline{d}%
		# # # # # # 
		#         #
		#     @   #
		#         #
		#         #
		# # # # # #
		
		Prossima posizione: %\underline{d}%
		# # # # # # 
		#         #
		#       @ #
		#         #
		#         #
		# # # # # #  
		
		Prossima posizione: %\underline{d}%
		# # # # # # 
		#         #
		# @       #
		#         #
		#         #
		# # # # # #  
	\end{lstlisting}
    
    \section{Risorse}
    \label{ch_res}
    
    \begin{lstlisting}[style=mycpp]
	#include <iostream>
		
	using namespace std;
		
	const int R = 4;
	const int C = 4;
		
	// Imposta il valore dell'elemento della matrice `A`
	// in posizione `(i,j)` al valore di `c`.
	void set(char A[], char c, int i, int j) {
	    A[i * C + j] = c;
	}
		
	// Restituisce l'elemento in posizione `(i,j)` della matrice `A`.
	char get(char A[], int i, int j) {
	    return A[i * C + j];
	}
		
	// Disegna la matrice `A`.
	void disegna(char A[]) {
	    for (int i = 0; i < C + 2; i++) cout << "#" << " "; 
	    cout << endl;
	    for (int i = 0; i < R; i++) {
	        cout << "#" << " ";
	        for (int j = 0; j < C; j++) {
	            cout << A[i * C + j] << " ";
	        }
	        cout << "#" << endl;
	    }
	    for (int i = 0; i < C + 2; i++) cout << "#" << " ";
	    cout << endl;
	}
		
	// Inizializza la matrice `A` con il carattere `c`.
	void inizializza(char A[], char c) {
	    for (int i = 0; i < R; i++) {
	        for (int j = 0; j < C; j++) {
	            set(A, c, i, j);
	        }
	    }
	}

	    // **************************************
	int main() {
	    char A[C * R];
		
	    inizializza(A, ' ');
	    set(A, '@', 0, 0);
		
	    // DA FARE: animare la matrice e aggiornarla in base alla
	    // direzione in cui l'utente desidera andare
		
	    // Hint: chiedere all'utente la direzione in cui muovere lo snake.
	    //       Provare inizialmente muovendo uno snake di lunghezza 1.
		
	    disegna(A);
		
	    return 0;
	}
    
    \end{lstlisting}
    
\end{document}