%% ***************************************************************
%  Copyright (C) Luca Parolari 2020
%  
%  luca.parolari23@gmail.com
%
%  You should have received a copy of the license with this file,
%  if not write to the author and request the license.

% !TeX spellcheck = it_IT

%\documentclass[addpoints,12pt]{exam}
\documentclass[addpoints,12pt,answers]{exam}

%% ***************************************************************
%  PACKAGES
%  ========
\input{packages.tex}

%% ***************************************************************
%  RESOURCES
%  =========
\input{prooftree.tex}
\input{macros.tex}
\input{config.tex}

%% ***************************************************************
%  CONFIGURATIONS
%  ==============
\input{exerciseconfig.tex}


% ****************************************************************
% DOCUMENT
% ========

\author{Luca Parolari\footnote{\href{mailto:luca.parolari23@gmail.com}{luca.parolari23@gmail.com}}}

\begin{document}

\title{Cassaforte}
\date{AS 2019-2020}

\maketitle

Leggere attentamente la consegna e svolgere l'esercizio.

\section{Consegna}
Progettare e scrivere un programma che simula il comportamento di una cassaforte.

La cassaforte, rappresentata da una classe, deve essere costruita di modo che possa avere
uno stato interno impostato alla creazione dell'oggetto. In particolare la cassaforte
memorizza una \textit{password} (cifrata con un algoritmo anche molto semplice) e
un \textit{segreto} che rappresentata il contenuto della cassaforte. Inoltre la
cassaforte può trovarsi in stato bloccato o sbloccato.

La cassaforte presenta anche una serie di metodi per l'interazione:
\begin{itemize}
    \item \texttt{sblocca(pwd)}, sblocca la cassaforte controllando che la password fornita sia la stessa di quella impostata
    \item \texttt{blocca()}, blocca la password
    \item \texttt{segreto()}, restituisce (o stampa) il segreto a video, ma solo se la cassaforte è stata sbloccata con la password corretta
    \item \texttt{cambiaSegreto(nuovoSegreto)}, cambia il segreto custodito nella cassaforte
    \item \texttt{cambiaPassword(nuovaPwd)}, cambia la password solo se la cassaforte è sbloccata
\end{itemize}

Realizzare un main interattivo tramite il quale l'utente può interagire con la cassaforte.
La cassaforte iniziale è costruita con dei valodi di default arbitrariamente scelti dal programmatore.

\newpage

\section{Esempi di utilizzo}
\footnote{Input sottolineato.}

\begin{lstlisting}[style=verbatim]
> ./a.out
Operazioni disponibili: [
    1=sblocca, 
    2=blocca,
    3=segreto,
    4=cambia segreto,
    5=cambia pw,
    0=esci ]

Scegli un'operazione: %\underline{1}%
Inserisci password: %\underline{test}%
Password errata, riprovare

Scegli un'operazione: %\underline{1}% 
Inserisci password: %\underline{admin}%
Cassaforte sbloccata

Scegli un'operazione: %\underline{3}%
Ecco il segreto: il mio tessoro

Scegli un'operazione: %\underline{2}%
Cassaforte bloccata

Scegli un'operazione: %\underline{4}%
Devi prima sbloccare la cassaforte

Scegli un'operazione: %\underline{5}%
Devi prima sbloccare la cassaforte

Scegli un'operazione: %\underline{0}%
Arrivederci
\end{lstlisting}

\end{document}