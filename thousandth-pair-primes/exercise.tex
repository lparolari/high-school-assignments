%% ***************************************************************
%  Copyright (C) Luca Parolari 2020
%  
%  luca.parolari23@gmail.com
%
%  You should have received a copy of the license with this file,
%  if not write to the author and request the license.

% !TeX spellcheck = it_IT

%\documentclass[addpoints,12pt]{exam}
\documentclass[addpoints,12pt,answers]{exam}

%% ***************************************************************
%  PACKAGES
%  ========
\input{packages.tex}

%% ***************************************************************
%  RESOURCES
%  =========
\input{prooftree.tex}
\input{macros.tex}
\input{config.tex}

%% ***************************************************************
%  CONFIGURATIONS
%  ==============
\input{exerciseconfig.tex}


% ****************************************************************
% DOCUMENT
% ========

\author{Luca Parolari\footnote{\href{mailto:luca.parolari23@gmail.com}{luca.parolari23@gmail.com}}}

\begin{document}
    
    \title{Millesima Coppia Numeri Primi}
    \date{08/02/2020}
    
    \maketitle
    
    Leggere attentamente la consegna e svolgere l'esercizio.
    
    \section{Consegna}
    Un numero si dice primo se e solo se è divisibile solamente per 1 e per se stesso.

    Data la sequenza di numeri primi
    \[ 
        2, 3, 5, 7, 11, 13, 17, 19, 23, 29, 31, 37, 41, 43, ...
    \]
    29 e 31 sono la quinta coppia di numeri primi più vicini tra loro, ovvero li separa un solo numero, il 30.
    Qual è la 1000-esima coppia di numeri primi separati solo da un numero?
    
    Progettare e scrivere un programma che risolve il problema sopra descritto.
    
\end{document}