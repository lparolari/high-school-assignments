%% ***************************************************************
%  Copyright (C) Luca Parolari 2020
%  
%  luca.parolari23@gmail.com
%
%  You should have received a copy of the license with this file,
%  if not write to the author and request the license.

% !TeX spellcheck = it_IT

%\documentclass[addpoints,12pt]{exam}
\documentclass[addpoints,12pt,answers]{exam}

%% ***************************************************************
%  PACKAGES
%  ========
\input{packages.tex}

%% ***************************************************************
%  RESOURCES
%  =========
\input{prooftree.tex}
\input{macros.tex}

%% ***************************************************************
%  CONFIGURATIONS
%  ==============
\input{exerciseconfig.tex}


% ****************************************************************
% DOCUMENT
% ========

\author{Luca Parolari\footnote{\href{mailto:luca.parolari23@gmail.com}{luca.parolari23@gmail.com}}}

\begin{document}
    
    \title{Biblioteca\\ \large Esercizio di Programmazione}
    \date{Dicembre 2019}
    
    \maketitle
    
    Leggere attentamente la consegna e svolgere l'esercizio rispettando il più possibile le specifiche date per quanto riguarda i metodi della classe e i tipi di dato in gioco.
    
    \section{Consegna}
    Si vuole realizzare un software per la gestione di una piccola biblioteca. Il software dovrà permettere all'utente di inserire nuovi libri nella biblioteca e cercarli per autore.
    
    Si progetti una classe \texttt{Libro} che permetta di memorizzare in modo privato le informazioni sul titolo, l'autore e l'anno di pubblicazione del libro. Il costruttore della classe obbliga l'utente ad inserire tutte le informazioni sul libro. Inoltre, la classe \texttt{Libro} espone i metodi \texttt{nome()}, \texttt{autore()} e \texttt{anno()} che restituiscono i rispettivi campi dato, mentre non sono forniti metodi per la modifica delle informazioni del libro. Si aggiunga anche un metodo \texttt{to\textunderscore string()} che restituisce una stringa composta da \texttt{NOME\textunderscore LIBRO, AUTORE, ANNO} del libro.
    
    Si progetti una classe \texttt{Biblioteca} che permette di memorizzare in modo privato un array di libri. La classe fornisce il metodo \texttt{aggiungi(Libro l)} che consente di aggiungere un nuovo libro alla biblioteca (lo inserisce nell'array), un metodo \texttt{vedi()} che stampa a video tutti i libri nella biblioteca e un metodo \texttt{cerca(string autore)} che stampa a video il \emph{primo} risultato che coincide con la ricerca.
    
    \section{Dettagli implementativi}
    Si utilizzino tutti gli strumenti forniti dal C++ 11. Potranno essere utili le seguenti astrazioni ed operazioni.
    \begin{itemize}
    	\item \texttt{vector}: astrazione dell'array. Utilizzabile come in C, ma fornisce in più metodi come \texttt{push\textunderscore back} per l'inserimento di elementi in coda. Non necessita di una dimensione fissata. Istanziare il vettore come un array di libri, ad esempio
\begin{verbatim}
vector<Libro> v;
v.push_back(l); // l è un libro
\end{verbatim}

		\item \texttt{string}: astrazione sulle stringhe. Tornerà utile il metodo \texttt{find(string s)} (documentazione \href{http://www.cplusplus.com/reference/string/string/find/}{http://www.cplusplus.com/reference/string/string/find/}); ad esempio
\begin{verbatim}
if (l.autore().find(autore_da_cercare) != string::npos) {
    // la stringa `autore_da_cercare` è una sottostringa del valore
    // ritornato dal metodo `autore()` chiamato sul libro.
}
\end{verbatim}

		\item \emph{foreach}, un costrutto che permette di iterare su tutti gli elementi dell'array, ad esempio
\begin{verbatim}
// v è un vettore di libri
for (Libro l : v) {
    // utilizzo l (oggetto di tipo Libro)
    cout << l.to_string() << endl;
}
\end{verbatim}
    \end{itemize}
    
    
    \section{Esempi di utilizzo}
    
\begin{lstlisting}[style=mycppbox]
int main() {
    Libro l1("C++. Linguaggio, libreria standard, principi di programmazione", "Bjarne Stroustrup", 2015);
    Libro l2("Clean Code: A Handbook of Agile Software Craftsmanship", "Robert C. Martin", 2008);
    Libro l3("Programming in Haskell", "Graham Hutton", 2016);

    Biblioteca b;

    b.aggiungi(l1);
    b.aggiungi(l2);
    b.aggiungi(l3);

    cout << "BIBLIOTECA" << endl;
    b.vedi();
    cout << endl;

    cout << "CERCA UN LIBRO" << endl;
    b.cerca("Graham");
    b.cerca("o");
    b.cerca("Pippo");

    return 0;
}

/* **************************************
   OUTPUT
   **************************************

BIBLIOTECA
C++. Linguaggio, libreria standard, principi di programmazione, Bjarne Stroustrup, 2015
Clean Code: A Handbook of Agile Software Craftsmanship, Robert C. Martin, 2008
Programming in Haskell, Graham Hutton, 2016

CERCA UN LIBRO
Programming in Haskell, Graham Hutton, 2016
C++. Linguaggio, libreria standard, principi di programmazione, Bjarne Stroustrup, 2015

*/
\end{lstlisting}
    
\end{document}