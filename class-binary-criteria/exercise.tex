%% ***************************************************************
%  Copyright (C) Luca Parolari 2020
%  
%  luca.parolari23@gmail.com
%
%  You should have received a copy of the license with this file,
%  if not write to the author and request the license.

% !TeX spellcheck = it_IT

%\documentclass[addpoints,12pt]{exam}
\documentclass[addpoints,12pt,answers]{exam}

%% ***************************************************************
%  PACKAGES
%  ========
\input{packages.tex}

%% ***************************************************************
%  RESOURCES
%  =========
\input{prooftree.tex}
\input{macros.tex}
\input{config.tex}

%% ***************************************************************
%  CONFIGURATIONS
%  ==============
\input{exerciseconfig.tex}


% ****************************************************************
% DOCUMENT
% ========

\author{Luca Parolari\footnote{\href{mailto:luca.parolari23@gmail.com}{luca.parolari23@gmail.com}}}

\begin{document}
    
    \title{Classe Criterio Binario}
    \date{AS 2019-2020}
    
    \maketitle
    
    Leggere attentamente la consegna e svolgere l'esercizio.
    
    \section{Consegna}

    Progettare una classe \texttt{CriterioBinario} che viene costruita prendendo in input una stringa che rappresenta un operatore binario e un numero, ad esempio \texttt{CriterioBinario c("<", 3)}. Se l'operatore binario non è supportato, il criterio risulta sempre falso.

    La classe \texttt{CriterioBinario} mette a disposizione un metodo \texttt{vale(numero)} che restituisce vero o falso se il criterio memorizzato vale sul numero dato in input.

    Ad esempio:
    \begin{lstlisting}[style=mycpp]
    CriterioBinario criterio("==", 5);
    criterio.vale(7) // restituisce false perche 7 != 5
    criterio.vale(5) // restituisce true perche 5 == 5
    \end{lstlisting}
  
    Scrivere poi due funzione esterne alla classe CriterioBinario, dove
    \begin{itemize}
        \item la prima prende in input un criterio ed un array di numeri e verifica se il criterio vale su tutti i numeri dell'array
        \item la seconda prende in input un criterio ed un array di numeri e verifica se il criterio vale su almeno un numero dell'array
    \end{itemize}

    Testare le funzioni realizzate nel main, ad esempio
    \begin{lstlisting}[style=mycpp]
    int arrayInteri[] = {1,2,3,4,5,6,7,8};
    CriterioBinario minoreDi3("<", 3);
    
    tuttiVeri(minoreDi3, arrayInteri) // restituisce false
    unoVero(minoreDi3, arrayInteri) // restituisce vero
    \end{lstlisting}
    
\end{document}