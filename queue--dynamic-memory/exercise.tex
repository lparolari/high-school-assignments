%% ***************************************************************
%  Copyright (C) Luca Parolari 2020
%  
%  luca.parolari23@gmail.com
%
%  You should have received a copy of the license with this file,
%  if not write to the author and request the license.

% !TeX spellcheck = it_IT

%\documentclass[addpoints,12pt]{exam}
\documentclass[addpoints,12pt,answers]{exam}

%% ***************************************************************
%  PACKAGES
%  ========
\input{packages.tex}

%% ***************************************************************
%  RESOURCES
%  =========
\input{prooftree.tex}
\input{macros.tex}
\input{config.tex}

%% ***************************************************************
%  CONFIGURATIONS
%  ==============
\input{exerciseconfig.tex}


% ****************************************************************
% DOCUMENT
% ========

\author{Luca Parolari\footnote{\href{mailto:luca.parolari23@gmail.com}{luca.parolari23@gmail.com}}}

\begin{document}
    
    \title{Tipo di dato astratto coda\\ \large (con memoria dinamica)}
    \date{Novembre 202}
    
    \maketitle
    
    Leggere attentamente la consegna e svolgere l'esercizio.
        
    \section{Consegna}
    
    Si realizzi un programma che implementa il tipo di dato astratto
    coda. Una coda o \emph{queue} mette a disposizione, tra le varie
    funzioni di utilità che andranno definite, due operazioni 
    principali: \textbf{enqueue} (mettere in coda, sempre dalla testa)
    e \textbf{dequeue} (rimuovere dalla coda, sempre dal fondo).

    Il comportamento della coda permette di gestire il contenitore 
    di oggetti proprio come una coda al supermercato: il primo che 
    arriva è il primo ad essere servito ed è quindi il primo a essere
    rimosso dalla coda. Gli altri, arrivati successivamente,
    si accodano e vengono estratti in ordine di arrivo. Analogamente,
    nel mondo informatico possiamo vedere la coda come un buffer FIFO
    (Fist In First Out).

    Tra le diverse implementazioni possibili, si richiede di
    utilizzare una \emph{linked list} come contenitore di base per le
    funzionalità della \emph{queue}.  I metodi a disposizione sull'ADT
    \emph{queue} andranno quindi a sfruttare le funzionalità della
    \emph{linked list} e a mascherare tutte le operazioni che la coda
    non deve fornire. Per esempio, non deve essere possibile inserire
    un elemento nel mezzo della coda!

    La coda deve quindi fornire i seguenti metodi.

    \begin{itemize}
        \item Un metodo \emph{inserisci} (rappresenta la funzionalità di
          \emph{enqueue}) che inserisce gli elementi sempre dal fondo
          della coda.
        \item Un metodo \emph{togli} (rappresenta la funzionalità di
          \emph{dequeue}) che rimuove gli elementi dall'inizio della
          coda.
        \item Un metodo \emph{primo} che permette di visualizzare il
          primo elemento (in testa) della coda.
        \item Un metodo \emph{is\_vuota} che restituisce \texttt{true}
          se la lista è vuota, \texttt{false} altrimenti.
        \item Un metodo \emph{costruisci\_vuota} che costruisce una coda
          composta da $0$ elementi.
        \item Vari metodi definiti da utente (a piacere) che
          permettono di popolare una coda da altri contenitori, per
          esempio da un array.
        \item Un metodo \emph{dimensione} che restituisce la dimensione
          della coda.
        \item Altre funzioni di utilità opzionali definite 
          dall'utente coerenti con la filosofia della coda.
    \end{itemize}

    \noindent\textbf{Nota.} I dettagli sono appositamente non definiti.
    
    \vspace{1em}\noindent\textbf{Suggerimento.} La maggior parte dei metodi sopra
    citati possono essere facilmente implementati utilizzando i metodi
    della \emph{linked list} ($\leq 2$ righe di codice per
    metodo).

    \vspace{1em}\noindent\textbf{Estensione} \emph{(opzionale)}.  L'efficienza
    migliore in termini di implementazione si ottiene se vengono
    mantenuti i riferimenti sia alla testa che alla coda della lista,
    ovvero, se al posto di una \emph{linked list ``semplice''} si
    implementa una
    \emph{\href{https://it.wikipedia.org/wiki/Deque}{double ended
        queue}}.
    \footnote{\href{https://it.wikipedia.org/wiki/Deque}{https://it.wikipedia.org/wiki/Deque}} \\
    Modificare la \emph{linked list} per implementare una
    \emph{double ended queue} e rendere più efficiente
    l'implementazione della coda.

    \section{Esempi di utilizzo}

    Il seguente codice è un esempio di main che testa
    alcune funzionalità della coda. 
    
    \begin{lstlisting}[style=verbatim]
        // Esempio: cassa di un supermercato.

        Coda::Coda c1 = Coda::costruisci_vuota(); // fila alla cassa 1
        Coda::Coda c2 = Coda::costruisci_vuota(); // fila alla cassa 2

        c1 = Coda::inserisci(c1, 1); // arriva il cliente 1 e va alla cassa 1
        c2 = Coda::inserisci(c2, 2); // arriva il cliente 2 e va alla cassa 2
        c1 = Coda::inserisci(c1, 3); // arriva il cliente 3 e va alla cassa 1

        std::cout << Coda::dimensione(c1) << std::endl; // restituisce 2
        std::cout << Coda::dimensione(c2) << std::endl; // restituisce 1

        std::cout << Nodo::valore(Coda::primo(c1)) << std::endl; // restituisce 1 (il cliente 1)

        Coda::stampa(c1); // stampa [1, 3, ]
        Coda::stampa(c2); // stampa [2, ]

        c1 = Coda::togli(c1); // viene servito il cliente 1 della cassa 1
        c2 = Coda::togli(c2); // viene servito il cliente 2 della cassa 2

        Coda::stampa(c1); // stampa [3, ] che sono i clienti ancora in attesa alla cassa 1
        Coda::stampa(c2); // stampa [] che sono i clienti ancora in attesa alla cassa 2(nessuno !)

        // Visualizza il primo cliente della cassa c2 (se esiste!)
        if (!Coda::is_vuota(c2))
        {
          std::cout << Coda::primo(c2) << std::endl;
        }
        else
        {
          std::cout << "La lista e' vuota!" << std::endl;
        }

        c1 = Coda::inserisci(c1, 4); // arriva il cliente 4 e va alla cassa 1
        c2 = Coda::inserisci(c2, 5); // arriva il cliente 5 e va alla cassa 2

        Coda::stampa(c1); // stampa [3, 4, ]
        Coda::stampa(c2); // stampa [5, ]
    \end{lstlisting}
    
\end{document}
